Le HAUM est né en 2012 de la rencontre entre quelques étudiants (avec un projet de hackerspace) et LinuxMaine
(LUG\footnote{Linux User Group}). Le hackerspace s'est construit autour des envies de logicels libres, de compréhension
et de partage de chacun.

\section{Principes}

L'association est un hackerspace, c'est à dire un lieu de rencontre où chacun peut venir pour travailler sur ses projets
personels ou ceux de l'association. L'entraide est de mise et l'essai encouragé.

Un hackerspace est une \textit{do-ocracy}\footnote{Fait-cratie, celui qui fait est celui qui décide} et chacun des
membres prend part aux décisions, à la vie de l'association et à son évolution.

