De nombreux projets ont vu le jour au HAUM. Qu'ils soient musicaux, techniques, citoyens ou ludiques, tous sont open-source.

\section{Projets Artistiques/Musicaux}

\subsection{PianoStairs}
PianoStairs est un piano dont les touches sont faites pour etre posées sur les marches d'un escalier. Ces touches de piano sont reliées à un arduino qui se charge d'interpréter les appuis sur les touches pour les convertir en notes de musique au format MIDI, puis un PC, connecté à l'ensemble, s'occupe de convertir les notes de musiques en sons !

\subsection{dHAUMidi}
dHAUMidi et dHAUM sont deux projets complémentaires créés à l'ocasion du festival Teriaki en Aout 2015. Ce projet abouti à la création d'un dôme géodésique en bois (le dHAUM), surmonté d'une carte Léoké (une carte Makey-Makey modifiée) à laquelle sont connecté une multitude d'objets conducteurs (dHAUMidi). Le but étant de toucher deux objets, ou plus, pour créer une note de musique, et donc, de jouer de la musique de manière non-conventionnelle !


\section{Projets Lumineux}

% Doit on parler de aziPOV? Ce projet n'est actuellement pas construit et ne l'a jamais été à ce jour.
%\subsection{Azipov}

\subsection{Pong1D}
Pong1D, comme son nom l'indique, est un remake du jeu mythique Pong, mais sur une seule dimension. Le jeu se joue grace à un bandeau de LED qui est parcouru par la "balle" de jeu, et l'objectif est de parvenir à renvoyer cette balle à l'adversaire en appuyant sur le bouton de sa manette au bon moment. Dextérité et réflexes requis ! Tecchniquement, un arduino pilote la bande de LEDs et capte les pressions sur le bouton de chaque manette, le tout grâce à la bibliothèque PolychrHAUM !

\subsection{HAUMTinsel}
La période de Noël approchant, le Pong1d s'est fait hacké pour le transformer en guirlande de Noël intéractive, sous la forme d'un jeu accessible sur Internet. La guirlande est connectée à une carte arduino qui contient un code basé sur PolychrHAUM, la aussi. Une partie du code est hébergé sur un "serveur", une raspberry pi dans notre cas, sert les pages aux différents joueurs. L'objectif du jeu est de cliquer sur le Père-Noël qui se déplace (vite !), pour ajouter une LED de la couleur de l'équipe du joueur à la guirlande, et espérer remplir la guirlande de cette même couleur.

\subsection{PolychrHAUM}
PolychrHAUM est un projet qui a pour but de faciliter l'utilisation de bandeaux de LEDs.

\section{Projets d'Outillage}

\subsection{PCBlastifieuse}


\subsection{AxiHAUM}

\section{Projets Citoyens \& Projets de réflexion}

\subsection{Opendata}

\subsection{Flexagones}
