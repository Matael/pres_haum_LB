\section{Matériel disponible}

Au fil des mois, le hackerspace a pu récupérer un certain nombre d'outils et une quantité intéressante de matières
premières.

Parmi les outils, on pourra citer notamment :

\begin{description}
    \item[Fraiseuse numérique 3 axes] Nommé AxiHAUM, cette fraiseuse est née de la modification d'une fraiseuse à PCB
        des années 90. De 2 axes, elle est passée à 3 et est aujourd'hui pilotée \textit{via} LinuxCNC.
    \item[Plastifieuse augmentée] La PCBlastifieuse est une plastifieuse augmentée capable de monter suffisament en
        température pour pouvoir transférer du toner sur une plaque de cuivre. Cet outil permet la réalisation de PCB de
        bonne qualité à moindres frais.
\end{description}

Le HAUM dispose aussi d'un serveur.




